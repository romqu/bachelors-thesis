\subsection{Problem}
\label{subsec:problem}

Today's applications has been increasing in complexity over the last few years.
Notably in the field of frontend development, the amount of functions has increased \cite{kevin2018}. The web development has transitioned from server rendered, 
page-reloading websites, to modern so called single page applications or SPA's.
The same applies to the mobile world: social networks, navigation, sharing and editing files together is a commonly demanded by the user.
A large part of applications are thus in exchange with APIs, interacting with local or external databases
and communicating with the underlying operating system itself (eg the periodic recording of the location via GPS or WiFi).
The challenge a developer faces is to come up with a general approach that bridges the gap between what the users sees, the source code
and the outside world - mainly: How to structure the source code. Ähnlich wie bei vielen Problemen die in der Softwareentwicklungen bestehen, existiert 
auch dieses seit geraumer Zeit und betrifft einen Großteil der Entwickler. Es is somit eine immer wiederkehrendes Problem. Daraus ensteht häufig das Bestreben, 
allgemeingültige Konzepte zu entwickeln, die dieser Problematik entgegenwirken und einen möglichen Lösungsweg aufzeigen. Die hier gesuchten Konzepte werden
als sogenannte "Entwurfsmuster" deklariert. Ihr zielt es, den Aufbau bzw. die Architektur von Quellcode so zu gestalten, dass er modular, flexibel und
wiederverwendbar ist. Zeitgleich wird der Entwickler dazu gewzungen einem Schema zu folgen und konistent zu arbeiten. Dies fördert die Qualität und
Wartbarkeit der Applikation. Über die Zeit haben sich viele Entwurfsmuster für die Strukturierung von Code innerhalb der Präsentationsschicht entwickelt. 
Dazu gehören unteranderem, nach "Erscheinungsjahr" sortiert: Model-View-Controller (MVC), Model-View-Presenter (MVP) Model-View-ViewModel (MVVM) und - relativ 
neu im Bunde - Model-View-Intent (MVI). Jedes der hier aufgezählten Muster dient dabei dem gleichen Zweck: Die strikte Trennung der Benutzeroberfläche von der ihr 
zugrundeliegenden (Geschäfts) Logik.
Auch MVI bedient sicher dieser Idee. In das Leben gerufen von André (Medeiros) Staltz, sieht es den Nutzer als Funktion.