Um den genannten Problem entgegenzuwirken gilt es, ein Muster zu entwerfen,
in welchem der "Zustand" das Hautaugenmerk bildet.
Ein gute Methode um mit Zuständen zu arbeiten, bietet das Konzept der endlichen Automaten.
Hierbei wird ein Startzustand festgelegt, und eine endliche Anzahl von Zuständen, die im System mit
Zustandsübergängen erreicht werden können. Wichtig ist, dass sich das System immer nur in genau
einem Zuständ befindet.
Im Kern befindet sich ein endlicher Automat, ein sogenannter Mealy-Automat.