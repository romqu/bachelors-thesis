
% Mention hold and cold observables
% Chain
% Backpressure
% talk about expression vs statements ?

% Asynchronicity -> Reactive -> Rxjava
% check if we actually say this

% 4. good for side effects ?
% 5. helps to avoid global state ?

% Copy-on-write / immutability A technique that blends the advantages of mutable and immutable objects

% Hypothesis
In order to reduce the cognitive burden on the developer of managing the different conditions and 
states the application can end up in, a small and lightweight, but opinionated library will be developed
in order to test this hypothesis.

% Intro, library core, functional 
The library tries favouring a declarative programming paradigm for its public api.
The core of the library will consist of a "Mealy" state machine. 
Thus the output of this finite state machine is determined by both its current state and inputs.
The input reflects an "event" that took place in the user facing layer.
That "event" is flexible and can be directly caused by the user (and his interactions) but also invoked
by the system directly as part of an initialisation flow. % \{todo}(or even by layers below )
Since a state machine is often defined purely in terms of mathematical functions with no reference to mutable state, 
the library will heavily rely on functional programming that embodies these concepts.

Internally pure functions will be used whenever it's possible as a methodology to adhere a state machines functionality.

In addition to that immutability will play a significant role. It makes data structures read only, limits them in their functionality 
and thus removes possible and unwanted side effects (Copy-on-write). Thereby it also makes multi-threading "safer" or "thread-safe".
The "state" data structure will serve as the single source of truth.

Another important part is to handle exceptions effeciently. Often they are used to control the flow of
the application, which can be viewed as an anti-pattern \cite{dontUseExceptionsForFlowControl}.
Although it will be up to the developer to decide what is considered a(n) exception/error/failure, 
the library will try to help to make it more convenient to handle.
For this task, the library again will use a concept of functional programming called "monads".
The exception handling is facilitated by the use of the so called 'Either' monad.
This allows a function to return either a left value or a right value, but only one exclusively at
any given time. By convention but not limited to, the left side is generally the negative case as in a failure or exception
and the right side represents the positive case.

To address the problem of todays application being inherently asynchronous and (therefore/often) developed 
with the use of multi threading, another abstraction layer is offered.
That layer is build upon reactive programming. It's a declarative programming paradigm
build on top of functional programming.

% Belongs into problem?
Asynchronous programming can result in a so called "callback hell", meaning that the developer must
explicitly handle each callback once a asynchronous task completes, which becomes cumbersome once the application 
and complexity grows and makes it harder to reason about. For this reason there exists a number of solutions
in most programming languages, for example "async/await" from Javascript or "coroutines" in the Kotlin world. 
But they all suffer from the fact that the streams in these frameworks ends after one event.

It's better to have streams that can fire multiple events before closing since it fits better into the
interactiveness of modern applications .

This is where the concept of Reactive programming comes in which takes a different approach. 
It wraps asynchronous tasks (inside of an observable) and treats everything as a stream of data.
These streams can easily be combined and transformed by applying functions on this data.

This makes it possible to listen to events instead of a classic polling based techniques to check
whether data is available. This behaviour is called the "Observer Pattern" and it follows the Hollywood Principle 
of "Don't call us, we'll call you".

The state of the application should be observable and the library should provide
a solution for asynchronous actions (side effects) out of the box. The developer
should work on streams, rather ...

https://gist.github.com/staltz/868e7e9bc2a7b8c1f754

As said by Andreas Staltz \ref{https://gist.github.com/staltz/868e7e9bc2a7b8c1f754#why-should-i-consider-adopting-rp} on
reactive programming:

"Reactive Programming raises the level of abstraction of your code so you can 
focus on the interdependence of events that define the business logic, 
rather than having to constantly fiddle with a large amount of implementation details."

The library must be deterministic and follow the principles of an unidirectional data flow. 
It will force the developer to create something that comes close to a "State transition table" and makes him 
think about every state the application can possibly be in.


% * pattern matching unterschuchen
% * strom von events

% * zustandsorientiere Modell -> inwieweit eignen sich reactive und functional?
% * wie lässt sich ein ereignis orientiertes Modell mit rfp umsetzen?
% * ist fp nützlich