Um den genannten Problem entgegenzuwirken gilt es, ein Muster zu entwerfen,
in welchem der "Zustand" das Hautaugenmerk bildet.
Ein gute Methode um mit Zuständen zu arbeiten, bietet das Konzept der endlichen Automaten.
Hierbei wird ein Startzustand festgelegt, sowie eine endliche Anzahl von Zuständen, die im System mit
Zustandsübergängen erreicht werden können. Wichtig ist, dass sich das System immer nur in genau
einem Zuständ befindet und dieser den SSOT (single source of truth) darstellt.
Kurz um: Der Automat befähigt einen, alle Zustände in dem sich ein System befinden kann
konsisten zu beschreiben. 
Unter den unterschiedlichen endlichen Automaten fällt die Wahl auf den sogenannten Mealy-Automat.
Bei diesem wird der nächste Zustand auf Basis des derzeitigen Zustands und Eingabe ermittelt.
Die Eingabe spiegelt dabei ein Ereigniss wieder, welches durch den Nutzer selbst (und seinen Interaktionen) 
oder durch Komponenten der Peripherie ausgelöst wurde. Zu nennen wäre bswp. 
der Klick auf einen Knopf oder die Mitteilung des Betriebssystem, das der Standort sich verändert hat.
Ein endlicher Automat wird meist durch mathematische Funktionen ausgedrückt, die 