\section{Anforderungsanalyse}
\label{sec:anforderungsanalyse}

\subsection{Funktionale Anforderungen}
Die Funktionalen Anforderungen dienen dafür, um die genaue Funktionalität und das Verhalten eines Systems zu beschreiben. Es wird behandelt, was das System können soll und muss. Um dies besser abbilden zu können, werden die einzelnen Anforderungen einer Gewichtung unterzogen. Diese Gewichtung lässt sich in Form von "Muss","Soll" und "Kann" Anforderungen ausdrücken. Aufgrund des kleines Rahmens und Zeitfensters dieser Thesis wird sich im diesen Teil ausschließlich auf "Muss"-Anforderungen beschränkt, d.h. jene Funktionalität, welches das Framework erfüllen muss. Für die Übersichtlichkeit werden sämtliche Anforderungen nummeriert und mit dem Kürzel "FA" versehen.
\\
\\
\textbf{[FA01] Identifizieren von "Intents"}
\\
Dem Nutzer des Frameworks muss es möglich sein, die "Intents" seiner Anwendung eindeutig zu markieren.
\\
\\


