\section{Grundlagen}
\label{sec:grundlagen}

In diesem Kapitel gilt es zu klären, auf welchen Grundlagen, Ideen und Konzepten Model-View-Intent beruht, wie diese miteinander fungieren und weshalb sie als Inspiration dienten.

\subsection{Unidirektionaler Datenfluss: Flux, Redux und React}

In einer Applikation existieren grundsätzlich zwei Komponenten: Eine, die der Nutzer wahrnehmen kann und eine, die für ihn unsichtbar bleibt. Bei ersterer handelt es sich meist um das, was der Nutzer(auf dem Bildschirm) sieht - die sogenannte »View«. Die zweite Komponente beschreibt die Ebene, welche das Geschehen observiert, darauf reagiert und den weiteren Verlauft (zum größten Teil) kontrolliert. Sie kann unter anderem als »Controller« betitelt werden.
\\
Der Zustand in dem sich eine Applikation befindet kann hierbei von beiden Seiten modifiziert und beobachtet werden. Ist dies der Fall, so handelt es sich um einen bidirektionalen Datenfluss. Bei dieser Variante entsteht die eventuelle Gefahr von kaskadierenden Updates als auch in einen unvorhersehbaren Datenfluss zu geraten. Des weiteren muss immer überprüft und sichergestellt werden, dass »View« und »Controller« synchronisiert sind, da beide den globalen Zustand darstellen. Schlussendlich verliert man zusätzlich die Fähigkeit zu entscheiden, wann der Zustand manipuliert wird.
\\
Ein anderer Ansatz ist, den Datenfluss in eine Richtung zu beschränken und ihn damit unidirektional
\cite{unidirectionalDataFlowFluxArchitectureIlyGelman2017, unidirectionalDataFlowTheCompleteReduxBookIlyGelman2017}
operieren zu lassen. Diese Variante erfreut sich an zunehmender Popularität seit der Bekanntmachung der »Flux«
\cite{fluxArchitectureAdamBoduch}
Architektur im Jahre 2015 von Facebook.
\cite{fluxAnnouncementYoutube}

\subsubsection{Flux}
In der »Flux« Architektur bilden zwei fundamentale Konzepte die Grundbaustein: 