\section{Fazit \& Ausblick}
Hier wird Resümee gezogen und in die Zukunft geschaut.

\subsection{Fazit}
Das Framwork besitzt ein paar gute Ansätze, schwächelt aber in der Umsetzung. Hierfür ist vor allem ein schlechtes Zeitmanagement und Trägheit des Verfassers schuld. Die Grundidee von MVI kommt im der Implementierung zur Geltung und der Entwickler welcher auf dieses Framework zurückgreift erhält einen gewissem Grad an Unterstützung bei der Anwendung vom MVI.
\\\\
Die größte Schwäche ist der Gebrauch von Reflexion. Sie mag in bestimmten Fällen Unabdingbar sein, kann in dieser Implementation allerdings an vielen Stellen durch Code Generierung substituiert oder gar komplett eliminiert werden. Ein Bespielt ist hierfür die Erstellung einer unveränderlichen Zustands Klasse auf Basis von Annotationen durch das Frameowork selbst. Damit fiele die Notwendigkeit für eine Überprüfung von diesem weg.
\\\\
Kurz um: Idee nett, Umsetzung mangelhaft.

\subsection{Ausblick}
Eine Weiterentwicklung und Nutzung des Framework ist durchaus denkbar. Hierfür müssen einige Verbesserungen wie die angesprochene Nutzung von Code Generation an Stelle von Reflexion vorgenommen werden. 