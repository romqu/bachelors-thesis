\section{Design \& Konzept}
\label{sec:design-und-konzept}
In diesem Kapitel ...

\subsection{Grundlegende Designentscheidungen}
Bevor auf Entscheidungen eingegangen wird ...

\subsubsection{Android als Plattform}
Das Framework richtet sich ausschließlich an Entwickler die Applikationen für die Plattform Android entwickeln. Es ist damit nicht kompatibel zu iOS, dem Web oder Serverseitigen Anwendungen. Die Spezialisierung lässt es jedoch zu, besser auf mögliche Eigenheiten der Plattform einzugehen. Ein weiterer Grund für diese Entscheidung stellt die Tatsache dar, dass MVI seinen Anfang in der Entwickelung von Webseiten fand und es sich im Fall Android um einen Nachzügler handelt.

\subsubsection{Framework}
Warum Framwork? Besonderheiten...

\subsubsection{Kotlin als Programmiersprache}
Die Applikationen in Android und das Android-SDK selbst sind bis vor wenigen Jahren fast ausschließlich in der Sprache Java entwickelt wurden. Seit der Google I/0 2017 gehört jedoch eine weitere Sprache zu den offiziell unterstützen:  Kotlin. Sie wird von dem Unternehmen Jetbrains entwickelt, die untere anderem die Entwicklungsumgebung Intellij für Java produzieren. Dieses bildet auch die Grundlage für Android Studio.
Kotlin hat in den letzten Jahren an Bodenhaftung gewonnen und findet auch intern bei Google Verwendung.
\\
Die Sprache wird als statisch typisierte, objektorientierte Programmiersprache bezeichnet und verfügt über eine hohe Interoperabilität zu Java. Dies bedeutet, dass innerhalb eines in Java geschriebenen Programms ohne viel Aufwand Kotlin genutzt werden kann. Dies ist ein wichtiger Faktor für die immer weiter ansteigende Beliebtheit, da es eine einfache Integration und bisherige Projekte gestattet. 
Kotlin bringt ein verbesserte Syntax mit und macht beispielsweise die Verwendung von "null" explizit.
Zu den Verbesserungen gehören dabei auch:
\\
\begin{itemize}
	\item Ableitung von Typen
	\item Alles ist eine Expression 
	\item Funktionen sind "First-Class-Funktionen" und bilden eine Funktionale Grundlage
	\item Datenklassen machen den Umgang mit unveränderliche Datenstrukturen einfach
	\item Erweiterungsfunktionen
	\item Kovarianz und Kontravarianz werden explizit anwendet
	\item Standardwerte für Parameter
\end{itemize}
\bigskip
\begin{lstlisting}[caption={Kotlin Beispiel}, label={lst:kotlin-beispiel}, language=Kotlin]
data class Example(
  privat val defaultMessage: String = "Hello World"
  privat val maybeNull: String? = null
){
  
  // Expression
  fun isHelloWorld() = when(message){
    "Hello World" -> true
    else -> false
  }
  
  // "?" findet bei null verwendung
  fun printIfNotNull() {
    maybeNull?.run { 
    	print(this)
    }
  }
}

// Erweiterungsfunktion und Funktion als Parameter
fun HelloWorld.extensionFunction(function: () -> String) { 
	val message = function()
	println(message)
}

// kein new Schlüsselwort nötig
// keine Semikolon nötig
val example = Example()

// copy wird automatisch generiert bei einer "data" Klasse
val newExample = example.copy(defaultMessage = "New Message")

// ist der letzte Parameter eine Funktion, so kann auf Klammern 
// verzichtet werden
newExample.extensionFunction { "Hello" }
\end{lstlisting}
\bigskip
Insgesamt is anhand Listing
\ref{lst:kotlin-beispiel}
zu erkennen, das Kotlin eine deutlich prägnantere und schlankere Syntax besitzt. Sie vermeidet damit einen großen Teil des mit Java verbundenen "Boilerplate-Codes" und kann für eine höheren Grad an Produktivität sorgen. Besonders der Umgang von Null als Teil des Typsystems kann vor der berühmten "Nullpointer-Exception" retten. Des weiteren besteht ein größerer Fokus auf dem Einsatz von Konzepten aus der funktionalen Programmierung, welche durch Erweiterungsfunktionen für bswp. Listen zum Einsatz kommen.

\subsection{Intent}
Jeder Intention geht ein Ereignis voraus, das entweder vom Nutzer oder der Anwendung selbst initiiert wurde. Es stellt dabei den Einstieg in den von MVI definierten Kreislauf (aus Abbildung) dar. 
\\
Klickt der Nutzer in einer Anwendung auf einen "Zurück" Knopf, so ist seine Intention zum vorherigen Bildschirm zurückzukehren oder die Anwendung zu beenden. Dieses Ereignis kann ohne weitere Informationen stattfinden. Anders ist es, wenn seitens des Nutzers innerhalb eine Liste ein Item ausgewählt wird und dessen Details gelistet werden sollen. Hierfür muss zusätzlich zu der eigentlichen Intention das ausgewählte Item (oder seine ID) übermittelt werden.
\\
Daraus ergeben sich zwei Arten von "Intents": Eines ohne und eines mit zusätzlichen Nutzdaten (Englisch payload). Dies bedeutet, dass eine Struktur existieren muss, die entweder Daten beinhaltet oder nur eine semantische Bedeutung hat.
\begin{lstlisting}[caption={Intent Klasse}, label={lst:intent-klasse}, language=Kotlin]
class Intent<T> (val payload: T)
\end{lstlisting}
\bigskip
Für diesen Fall eignet sich eine Klasse mit einem generischen Typ als Attribut, wie Listing
\ref{lst:intent-klasse}
zeigt. Das "<T>" in der Klassen Deklaration dient dabei als Platzhalter für den eigentlich Typ, z.B. für ein "Item" aus dem obigen Beispiel.
\\
Die aufgezeigte Option weißt jedoch gewisse Mängel auf:
\begin{enumerate}
	\item Der "Payload" darf niemals "null" sein
	\item Der Name "Intent" transportiert die eigentliche Absicht nicht
\end{enumerate}
\bigskip
Mangel Nr. 1 lässt sich mit einer einfachen Abwandlung von Listing
\ref{lst:intent-klasse}
behoben werden. 
\begin{lstlisting}[caption={Intent Klasse}, label={lst:intent-klasse-2}, language=Kotlin]
class Intent<T> (val payload: T? = null)
\end{lstlisting}
\bigskip
Hierfür muss lediglich von Kotlins Notation für "null"-Typen gebraucht gemacht werden: Das Fragezeichen. Um zu vermeiden, das bei dem erstellen einer Klasse ohne Inhalt stets "null" übergeben werden muss, wird ein Standardparameter verwendet. Listing
\ref{lst:intent-klasse-2}
stellt die Anpassungen dar.
\\
Für den zweiten Mangel gibt es unterschiedliche Ansätze:
\begin{enumerate}
	\item Ein zweites Attribut das die Absicht beschreibt
	\item oder eine eigene Klasse für jede Intention
\end{enumerate}
\begin{lstlisting}[caption={Intent Enum}, label={lst:intent-enum}, language=Kotlin]
enum class IntentDescription {
	GO_BACK, DISPLAY_ITEM_DETAILS
}
\end{lstlisting}
\bigskip
Bei Ansatz Nummer Eins ist ein potenzieller Kandidat die Verwendung eines "enum". Diese kann durch Konstanten (Listing 
\ref{lst:intent-enum} gibt einen Eindruck
) zum Ausdruck bringen, um welche Intention es sich handelt. Im nächsten Schritt muss das Enum als Attribut in der Intent Klasse hinterlegt werden:
\begin{lstlisting}[caption={Intent Klasse}, label={lst:intent-klasse-2}, language=Kotlin]
class Intent<T> (
  val payload: T? = null, 
  val description: IntentDescription
)

// die Explizite Angabe von Atrributsnamen
// bei weglassen von anderen Attrbuten ist
// best practice

val intent = Intent(description = IntentDescription.GO_BACK)
\end{lstlisting}
\bigskip
Aber auch diese Lösung ist suboptimal: Ungeachtet der Absicht ist immer ein "payload" von Nöten, selbst wenn dieser für das weitere Vorgehen nicht verwendet wird. Dies mag in wenigen Fälligen vertretbar sein, wird bei einer hören Anzahl an Intention unübersichtlich. Zusätzlich sollte immer versucht werden "null" Werten aus dem Weg zu gehen, wenn nicht zwingend erforderlich. Dies verringert die Gefahr unerwartet der "NullPointer Exception" über den Weg zu laufen.
\\
\\
Ein besser Ansatz bildet dabei Nummer zwei. Anstatt mit einer Klasse sämtliche Intentionen abbilden zu wollen, erscheint es sinnvoller für jede Intention eine Klasse zu kreieren. Bei dieser Variante ergeben sich zwei Eigenschaften: Jeder dieser Klassen stellt übergeordnet einen "Intent" dar und enthält möglicherweise einen "payload", welcher niemals "null" ist.
\\
Für dieses Szenario existiert ein Konzept das aus zwei Konstrukten hervorgehen kann. Das erste trägt den Namen "Produkt" und charakterisiert eine fundamentale Eigenschaft der Objekt Orientierten Programmierung. Es sagt aus, das mehrere unterschiedlichen Werte ein einzigen Wert bilden können. Darunter fällt bswp. eine Klasse in Java oder Kotlin.
\\
Das zweite Konstrukt, die Summe von Werten liegt vor, wenn anstatt von mehreren Werte zu einem entweder ein Typ oder ein anderer vorliegt. Es ist somit keine Kombination von Werten wie beim Produkt, sondern die Entscheidung für einen der Angegebenen. Das Enum wie in Listing
\ref{lst:summen-typ}
gehört unter anderem zu den Summen Typen.
\begin{lstlisting}[caption={Summen Typ}, label={lst:summen-typ}, language=Kotlin]
enum class Color(val rgb: Int) {
	RED(0xFF0000),
	GREEN(0x00FF00),
	BLUE(0x0000FF)
}

val color: Color = Color.RED

print(color is RED) // true
print(color is GREEN) // false
\end{lstlisting}
\bigskip
Vereint man beide Konstrukte, so ergibt sich die Idee eines algebraischen Daten Typen, der zusätzlich 
auch primitive Werte umfassen kann. Das Ziel ist, Daten die zusammengehören und einen gemeinsamen Nenner 
besitzen in einer übersichtlichen und transparenten Form darzustellen. Der Grund, warum diese Typen als 
"algebraisch" bezeichnet werden, ist, dass man neue Typen erschaffen kann, indem die "Summe" oder das 
"Produkt" bestehender Typen nimmt.
\\
\\
In Kotlin existiert für diesen speziellen Fall eine bestimmte Form der Klasse, die ihn ihrer Deklaration mit dem "sealed" Schlüsselwort eingeleitet wird. Dies macht es möglich, Listing 
\ref{lst:intent-klasse-2}
funktionaler und eleganter zum implementieren:
\begin{lstlisting}[caption={Intents als sealded class}, label={lst:intents-sealed-class}, language=Kotlin]
sealed class Intent {
	// 'object' erzeugt ein Singleton
	// es ist keine Instanziierung möglich
	// Attribute folglich nicht gestattet
	object GoBack : Intent()
	class DisplayItemDetails(item: Item): Intent()
}
\end{lstlisting}
\bigskip
Mit diesem Ansatz verschwindet die Notwendigkeit für einen generischen Typ und das Vorhandensein von 
"null" Werten. Des weiteren ist mit einem Blick erkennbar welche Intents vorkommen, sowie ihre Bedeutung 
und, wenn definiert, ihr Inhalt. Anders ausgedrückt dienen versiegelte Klassen zur Darstellung 
eingeschränkter Klassenhierarchien. Dann, wenn ein Wert einen Typ aus einer begrenzten Menge haben kann, 
aber keinen anderen Typ haben darf. Sie sind in gewisser Weise eine Erweiterung der Enum-Klassen: Der 
Wertebereich für einen Enum-Typ ist ebenfalls eingeschränkt, aber jede Enum-Konstante existiert nur 
einmal. Eine Unterklasse einer versiegelten Klasse kann derweil mehrere Instanzen haben, die überdies 
einen Zustand enthalten kann.
Zu beachten gilt außerdem, das in Kotlin sich dieses Konstrukt innerhalb einer Datei befinden muss.
\\
\subsection{View Model}
Die Erzeugung eines Intents findet entweder in einer "Activity", einem "Fragment" oder Derivaten statt. 
Infolge der Anlehnung von Model-View-Intent and Model-View-Controller muss eine Komponente existieren, 
welche die Business Logik innehat, die Intents entgegennimmt und sie im Kontext der Anwendung übersetzt. 
Weiterhin muss die Komponente reaktiv sein und den geforderten unidirektionalen Fluss einhalten.
\\
\\
Im ersten Schritt muss eine zentrale Funktion definiert werden, die als Parameter einen Intent erhält.
Diese muss vom Entwickler überschrieben und implementiert werden.
Zu diesem Zweck wird vom Konzept der Vererbung und dem oft damit verbundenen Polymorphismus (griechisch 
für Vielgestaltigkeit) Gebrauch gemacht. Erbt eine Klasse von einer anderen, so wird sie zu einer 
sogenannten Subklasse, während die andere eine Superklasse (auch Eltern- oder Basisklasse) darstellt. 
Sinn und Zweck der Vererbung ist die Wiederverwendbarkeit von Code und somit die Vermeidung von 
Redundanz. Es ermöglicht Code für Klassen die gewisse Merkmale teilen nur einmal schreiben zu müssen und 
diese dann davon ableiten zu lassen. Die Polymorphie liegt vor, wenn  zwei Klassen denselben 
Methodennamen verwenden, aber die Implementierung der Methoden sich unterscheidet.
Listing 
\ref{lst:poly}
verdeutlicht diesen Sachverhalt.
\begin{lstlisting}[caption={Vererbung \& Polymorphismus}, label={lst:poly}, language=Kotlin]
fun main(args: Array<String>) {

	val dog = Dog()
	val cat = Cat()

	// gleiche Methode, aber unterschiedliche Implementierung
	(dog as Animal).makeSound() // Wuff
	(cat as Animal).makeSound() // Miau
}

abstract class Animal {

	abstract fun makeSound()
}

class Dog : Animal(){

	override fun makeSound(){
		println("Wuff")
	}
}

class Cat : Animal(){

	override fun makeSound(){
		println("Miau")
	}
}
\end{lstlisting}
\bigskip
Für die Controller ähnliche Komponente bedarf es der Festlegung des vom Entwickler gewählten Intent 
Datentypen. In diesem Fall kann wieder auf den generischen Aspekt von Kotlin zurückgegriffen werden. 
Dieser Typ wird ebenfalls als Parameter Typ für die Intent-Funktion genutzt. Ein erster Entwurf kann wie 
folgt aussehen:
\\
\begin{lstlisting}[caption={Erster Controller Entwurf}, label={}, language=Kotlin]
abstract class Controller<I> {
	
	abstract fun intents(intent: Intent)
}

sealed class SomeIntent {
	// ...
}

class SomeCrontroller: Controller<SomeIntent> {

	override fun intents(intent: SomeIntent){
		// ....
	}
}
\end{lstlisting}
\bigskip
Im nächsten Schritt müssen die Intents einer Aktion zugeordnet bzw. in eine übersetzt werden. Dadurch, 
dass die Superklasse als Parameter verwendet, muss zu erst ermittelt werden, um welchen Subklasse von 
Intent es sich handelt. Für die Auswertung eines Intents kann auf eine weiteres, funktionales Konzept 
zurückgegriffen werden: Der Musterabgleich (Pattern Matching zu Englisch). Hierbei handelt es sich um ein 
Verfahren das prüft, inwieweit ein vorgegebenes Muster mit anderen "Formen"(hier Klassen) übereinstimmt. 
\\
\\
Auch in diesem Fall schafft Kotlin Abhilfe. Es stellt eine Expression bereit, welche eine elegante 
Handhabung für genau diesen Fall bietet. Sie ist dem "switch" Statement aus Java sehr ähnlich, offenbart 
aber mehr Möglichkeiten und ist dementsprechend mächtiger. Ein gewöhnliches switch Statement ist im 
Grunde nur eine Aussage, die eine Reihe von einfachen if/else ersetzen kann, welche grundlegende Prüfungen durchführen. Folgendes Vorteile kann mit der sogenannten "when" Expression erzielt werde:
\begin{enumerate}
	\item Es gibt immer einen Wert zurück (minimal "Unit"), ohne "return" angeben zu müssen
	\item Es ist kein Argument nötig
	\item Es kann beliebige Bedingungsausdrücke haben
	\item \label{when-design} Es hat ein allgemein besseres und sicheres Design (mögliche Überprüfung während der Kompilierung)
	\item \label{when-smart-cast} Es unterstützt "Smart Cast" oder "Auto Casting"
\end{enumerate}
Das nachfolgende Listing
\ref{lst:when-expression}
veranschaulicht die Anwendung dieser Expression mit mehreren Bedingungsausdrücken:
\begin{lstlisting}[caption={When Expression}, label={lst:when-expression}, language=Kotlin]
// Mit Argument
// Gibt einen String zurück
val result = when(number) {
	0 -> "Invalid number"
	1, 2 -> "Number too low"
	3 -> "Number correct"
	in 4..10 -> "Number too high, but acceptable"
	!in 100..Int.MAX_VALUE -> "Number too high, but solvable"
	else -> "Number too high"
}

// Ohne Argument
// Gibt ebenfalls einen String zurück
val res = when {
	x in 1..10 -> "cheap"
	s.contains("hello") -> "it's a welcome!"
	else -> ""
}
\end{lstlisting}
\bigskip
Die Punkte
\ref{when-design}
und
\ref{when-smart-cast}
sind dabei von besonderem Interesse. Durch die Verwendung einer "sealed" Klasse für die Implementierung der Intents ist es dem Compiler möglich festzustellen, inwieweit alle Subklassen innerhalb der "when" Expression abgedeckt sind. Fehlt eine, so wirft der Compiler einen Fehler und die Anwendung kann nicht ausgeführt werden. Des weiteren verliert der "else" seinen Nutzen, weshalb auf ihn verzichtet werden kann.
\\
Den nächsten Vorteil der sich ergibt, ist die Fähigkeit von Kotlin nach dem Bedingungsausdruck die Basisklasse automatisch in die korrekte Subklasse umzuwandeln. Dies wird als sogenanntes "Smart Casting" bezeichnet und erspart dem Entwickler ein manuelles umwandeln in die richtige Klasse.
Ein Vergleich zwischen Java und Kotin gibt Listing 
\ref{lst:when-smart-cast}
an.
\begin{lstlisting}[caption={"when" mit "smart cast"}, label={lst:when-smart-cast}, language=Kotlin]

// JAVA
val intent = Intent.GoBack

if(intent instanceof Intent.DisplayItemDetails){
	((Intent.DisplayItemDetails) intent).item // expliziter Cast
} else if(...) { 
	... 
}

// KOTLIN

// "is" ist Notwendig, sobald die Klasse Konstruktor 
// Argumente hat.
// Bei "object" is keines von Nöten, da die Klasse 
// nicht erzeugt wird.
when(val intent = Intent.GoBack){
	GoBack -> println("go back")
	is DisplayItemDetails -> 
		println(intent.item) // <-- Smart Cast
	// keine "else" Zweig erforderlich, da alle 
	// Subklassen abgedeckt sind
}
\end{lstlisting}
\bigskip
Mit diesem Ansatz lässt sich die Intent-Funktion umsetzen.

\subsection{Model \& Zustand}
Für die Repräsentation des Models gibt MVI an, das dieses Unveränderlich sein muss. Dies bedeutet, dass für jede Zustandsüberführung ein neues Model auf Basis des alten bzw. derzeitigen erzeugt werden muss. Dafür wird das Model kopiert und während diesem Vorgang mit neuen Werten versehen. 
\\
Auch für dieses Szenario stellt Kotlin eine Struktur zur Verfügung, welche den Prozess vereinfacht: Die "data" Klasse. Sie generiert mehrere Methoden welche dem Nutzer ohne weiteres Zutun zur Verfügung stehen. Darunter befindet sich unter anderem eine Methode die den Namen 'copy' trägt und wie der Name vermuten lässt, eine Kopie der Klasse erzeugt. Dabei können der Methode die Parameter übergeben werden, welche der Konstruktor inne hat. Dies macht es möglich, einzelne Attribute neu zu besetzten und bei anderen den Wert beizubehalten. Wie Listing
\ref{lst:data-class}
aufzeigt, gestaltet dies den Umgang mit Unveränderlichen Klassen verhältnismäßig einfach.
\begin{lstlisting}[caption={data class}, label={lst:data-class}, language=Kotlin]

data class MyClass(val text: String, val anotherText: String)

val myClass = MyClass("text", "anotherText")

// 'text' wird verändert, 'anotherText' nicht
val newMyClass = myClass.copy(text = "newText")
\end{lstlisting}
\bigskip
Eine Problematik die herbei jedoch besteht ist, dass bei der 'data' Klasse unveränderliche Attribute keine Pflicht sind. Somit wäre es dem Entwickler ohne weiteres Möglich, das Objekt direkt zu modifizieren, ohne ein neues zu erzeugen müssen. Um dies zu Verhindern gilt es Sicherzustellen, dass das vom Entwickler angedachte Model den Anforderungen entspricht. Für diesen Zweck kann aus zwei 'Werkzeugen' gewählt werden: Der Reflexion (oder Introspektion) und dem generieren von Code.  
\subsubsection{Reflexion (Introspektion)}
Bei dieser Variante ist es dem ausführenden Programm erlaubt seine eigene Struktur - zur Laufzeit - zu analysieren oder auch zu verändern (und Sichtbarkeitseinschrankungen zu umgehen). Es gestattet einem, Informationen über Klassen dynamisch auszulesen. Das beinhaltet Modifier, Variablen, Konstruktoren, Methoden, Annotationen (= reflektive Informationen) usw. Reflexion hat dabei vielfältige Anwendungszwecke:
\begin{enumerate}
	\item Debugger
	\item Interpreter
	\item Objekt Serialisation
	\item Dynamisches Laden von Code/ erzeugen von Objekten (z.B. Spring @Autowired)
	\item Java Beans
\end{enumerate}
\bigskip
In Kotlin muss beachtet werden, dass zwei Schnittstellen für den Einstieg in die Reflexion existieren. Einmal die Standard Schnittstelle für Java, und einmal jene für Kotlin. Erstere erlaubt die Arbeit mit allen Java Konstrukten und zweitere die, die Kotlin exklusiv sind. Für jede Klasse existiert zur Laufzeit ein Objekt des Typs 'Class<T>' oder KClass<T>. T ist dabei der Typ der zu untersuchenden Klasse.
\\
\\
Für den Zugriff auf die Java Reflexion muss auf die '.java' Endung zurückgegriffen werden, wie folgendes Beispiel verdeutlicht:
\begin{lstlisting}[caption={Java Reflexion}, label={lst:data-class}, language=Kotlin]

val myClass: Class<MyClass> = MyClass::class.java

// gibt Methoden wie 'toString' aus
myClass.methods.forEach(::println)
\end{lstlisting}
\bigskip
In Kotlin wird über den 'double-colon' Operator auf die Reflexion zugegriffen:
\begin{lstlisting}[caption={Kotlin Reflexion}, label={lst:data-class}, language=Kotlin]

val myClass: KClass<MyClass> = MyClass::class

// gibt den Namen der Klasse aus
println(myClass.qualifiedName)

// 'false', da 'text' nicht Konstant ist
println(myClass::text.isConst)
\end{lstlisting}
\bigskip
Eine weiterer Interessanter Ansatz ist die Arbeit mit sogenannten Metadaten. Sie stellen Information über Informationen dar. Ein häufiger Anwendungsfall ist die Arbeit mit Annotationen, welche mit '@' eingeleitet werden. So wird mit der 'override' Annotation dem Compiler in Java mitgeteilt, dass diese Methode überschrieben wurde:
\begin{lstlisting}[caption={Override Annotation}, label={lst:data-class}, language=Kotlin]

class MyClassJava {
	
	@Override
	public String toString() {...}
}

class MyClassKotlin {

// Hier ist 'override' teil der Deklaration
override fun toString() {...}
\end{lstlisting}
 

