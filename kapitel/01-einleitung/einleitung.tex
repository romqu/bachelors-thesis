\section{Einleitung}
\label{sec:einleitung}

Bis zum Jahre 1973 und der Entwicklung des ersten Eingabegerätes mit Graphischer Oberfläche (GUI) (Xerox Alto \cite{xeroxAlto}) erfolgte die Interaktion mit einem Computer im Wesentlichen über eine Konsole. Dies reduzierte die Ein- und Ausgabe eines Programms auf rein textuelle Elemente. Seither hat sich viel getan: Die Erschaffung des Internets leitet den Beginn von Webseiten ein, welche von einer anfänglich statischen Ausprägung zu der heutigen dynamisch und komplexen wuchsen. Dazu gesellten sich im Laufe der Zeit mobile Endgeräte - zu Beginn bestückt mit Tasten für die Eingabe, sowie einem primitiven Bildschirm für die Anzeige finden sich heutzutage vorwiegend leistungsfähige, auf einem kapazitiven Touchscreen basierende Smartphones wieder. Hierbei ist über die Jahre der Funktionsumfang von Betriebssystem und Applikationen im Allgemeinen gestiegen.


\subsection{Problemfeld}
\label{sec:problemfeld}
Die Herausforderung für einen Entwickler ist es, die Nutzerschnittstelle (auch: Presenation Layer \cite{presentationLayerpatternsOfEnterpriseApplicationArchitectureMartinFowler}) ,
innerhalb einer "Drei-Schichtenarchitektur"
\cite{threeTierArchitectureDonaldWolfe2013}
sinnvoll aufzubauen. Dabei müssen auch externe Vorkommnisse wie bswp. die Aktualisierung der zugrundeliegenden Datenbank berücksichtigt werden. Der Zustand welcher durch die Applikation an den Nutzer vermittelt wird, muss dabei in irgendeiner Art und Weise in dieser hinterlegt und verwaltet werden.
\\
Da diese Problematik nicht erst seit kurzem sondern seid vielen Jahren besteht, wurden hierfür 

\subsection{Ziel der Arbeit}
\label{sec:ziel-der-arbeit}

\subsection{Aufbau der Arbeit}
\label{sec:aufbau-der-arbeit}